%-*-coding:UTF-8-*-
%前沿结课作业
\documentclass[hyperref,UTF-8]{ctexart}
\newcommand{\cndash}{\rule{0.2em}{0pt}\rule[0.35em]{1.6em}{0.05em}\rule{0.2em}{0pt}}
\hypersetup{colorlinks=false,pdfborder=000}
\usepackage{geometry}
\renewcommand{\rmdefault}{ptm}
\geometry{a4paper,centering,scale=0.8}
\title{\heiti 地球物理在地质学中的应用}
\author{\kaishu}
\date{}
\usepackage{txfonts}
\usepackage{multirow,makecell}
\usepackage{fancyvrb}
\usepackage{pifont}
\begin{document}
\zihao{4}
\begin{titlepage}
\vspace*{40mm}
\begin{center}
{\heiti\Huge 地球物理学在地质学中的应用}\\[30mm]
{\Large 王亮}\\[5mm]
地质资源与地质工程\\\texttt{学号:102014079}\\[80mm]
2014年1月10日
\end{center}
\end{titlepage}
\maketitle
\tableofcontents
\begin{abstract}
这篇文档是《地球科学前沿》课的结课作业。本文先介绍地球物理学及其研究过程和特点,然后陈述地球物理在地质学应用的条件和方法选择的原则,最后举了三个例子:地球物理方法在地壳深部构造研究中的应用、综合地球物理方法在区域性地质填图中的应用和地球物理方法在工程探测与检测中的应用。
\end{abstract}
\section{地球物理介绍}
顾名思义,地球物理学是以地球为研究对象的一门应用物理学。
地球物理学运用物理学方法探测和研究地球内部以及地球表面附近的圈层结构、动力学过程、地球表面现象和深部结构及它们之间的联系。
这门科学包含许多分支学科,涉及海、陆、空三界,是天文、物理、化学和地质学之间的一门边缘科学。

地球物理学分为两部分:

\begin{enumerate}
\item 固体地球物理学
\begin{enumerate}
\item 普通地球物理学
\begin{itemize}
\itemsep=0pt \parskip=0pt
\item 研究大尺度现象和一般方法
\end{itemize}
\item 勘探地球物理(物理探矿)
\begin{itemize}
\itemsep=0pt \parskip=0pt
\item 勘探矿产
\end{itemize}
\end{enumerate}
\item 空间物理
\begin{itemize}
\item 研究高空乃至行星际空间
\end{itemize}
\end{enumerate}
\section{地球物理的研究过程和特点}
地球物理研究的过程分为三步:
\begin{enumerate}
\item 地球物理观测
\item 地球物理方法
\item 解释
\end{enumerate}
\subsection{地球物理观测}
地球物理,作为一门应用物理学,以观测为基础,按不同的观测技术和观测的地表物理量的不同可以进行分类:
\begin{center}
\begin{tabular}{c|c|c}
\hline
\hline
\multicolumn{3}{c}{不同的地球物理方法的测量}\\
\hline
观测技术&地表物理量&地内物理量\\
\hline
\multirowcell{3}{地震}&位移&地震波速度\\
&速度&品质因子\\
&加速度&密度\\
\hline
重力&重力加速度&密度\\\hline
\multirowcell{3}{地磁、地电}&磁感应强度&磁导率\\
&磁场方向&剩余磁化\\
&电位&电导率\\
\hline
热流&温度(梯度)&热流密度\\\hline
\hline
\end{tabular}
\end{center}

无论是做实际勘测工作还是理论研究,观测都是最重要的一环。理论研究再精细,没有观测做支柱也是空中楼阁。
\subsection{地球物理方法}
地球物理学是一门物理学,是以物理模型为研究的核心内容。
观测地球物理场,使用各种数理方法创建物理模型,然后计算物理模型的地球物理场,再和观测到的地球物理场进行对比
,验证物理模型是否符合事实,这是地球物理学的总方法。
根据物理模型计算地球物理场的过程称为正演;根据地球物理场反推物理模型称为反演。

地球物理得到的只有物理模型。通过地球物理,永远只能得到物性参数,而不可知是何物质,所以需要进行下一步解释。
\subsection{解释}
解释需要地质学、岩石学、矿物学等知识。
\subsection{特点}
地球物理是用物理模型来研究地学问题的。研究过程是数理计算,所以得到的结果非常准确。但是得到的仅是参数,不能据此知道是什么物质。
一些地球物理场来源于地球深处,故较由地表肉眼看到的情况推导地下结构的地质学,地球物理学能得到地球深部的信息。
\section{地球物理在地质学应用的概括}
地球物理较地质学的优势和劣势都很明显:地球物理能得到地球深部信息,且结果客观精确;但地球物理只能得到物性参数,何物质不可知。
地球物理在地质学中应用就要扬长避短,发挥地球物理的优势,回避它的劣势。
\subsection{地球物理的应用条件}
自然界中各种地质体均具有多种地球物理特征。这些地质体的条件必须符合一些特征,地球物理方法才能加以应用。
这些条件包括:

\begin{enumerate}
\item 目标体与围岩物理性质的差异程度,即对比度
\item 目标体的几何参数(形状、产状和规模)
\item 地球物理场的噪声水平
\end{enumerate}
\subsubsection{目标体与围岩物理性质的差异程度}
目标体和围岩的物理性质对任何一种地球物理方法都至关重要。不同的地球物理方法对不同的目标体与围岩间物理性质差异的要求各不相同。
所以,研究测区的岩矿石的物性参数就是开展物探工作所必须的了。获得这些参数一般有以下三种方法:
\begin{enumerate}
\item 收集物性参数的资料
\begin{itemize}
\item 使用前人资料最为直接、便捷。但岩石的物理性质受多种因素影响,使用资料时还需慎重。
\end{itemize}
\item 实验室测定
\begin{itemize}
\item 采集岩矿石,在实验室测量其物性参数。
\end{itemize}
\item 实地测量
\begin{itemize}
\item 在测区,选择露头好的地区实地测量。
\end{itemize}
\end{enumerate}
\subsubsection{目标体的几何参数}
待查的目标体的异常不仅取决于目标体与围岩的物性差异,也取决于其几何参数(规模、形状、产状及空间的相互位置)

目标体的埋深、规模与异常的大小有密切联系。比如均匀密度球体的重力加速度:
$$\Delta g=\frac{4{\piup}GR^3\sigma}{3H^2}$$

其中,$H$表示埋深。如果埋深很深,异常值就会很小。

目标体相对位置会影响地球物理方法的水平分辨率。当几个目标体靠得很近,相对距离会决定异常特征。
当几个目标体形成只有一个极值的整体异常时,对应的地球物理方法就无法把几个异常体区分开来。

目标体的形状和产状要素也具有很大意义。
比如地震勘探中,地质体的产状平缓比较有利。而电剖面法,陡倾界面比较有利。
\subsubsection{地球物理场的噪声水平}
地球物理测量仪器在接收目标体的信号的同时也接受到与目标体无关的干扰。
引起的干扰有三类:
\begin{enumerate}
\item 地质成因
\item 非地质成因
\item 测量误差
\end{enumerate}
地质成因的干扰是指地下围岩不均匀性引起的。最常见的是覆盖层、下伏岩层以及地形起伏的影响。

非地质成因的干扰则是由于一些地球物理场
本身在发生变化。比如接近磁极、磁场日变和磁暴等都会对磁力勘探有极大影响。除了自然因素也有可能是人为造成,比如:
电网的游散电流也随时间变化。

测量误差则分为系统误差和随机误差。
系统误差是由于仪器和测量方法的不完善引起的。随机误差则由很多因素引起。
\subsection{地球物理方法的选择原则}
地球物理方法的综合应用十分复杂。它不仅取决于要解决的地质任务,还需考虑到高效和低耗的要求。
在地质任务和总经费确定之后,究竟选用什么地球物理方法,哪些作为主流方法,哪些作为配合方法,
是地球物理测量工作设计的主要内容。
\section{地球物理方法在地壳深部构造研究中的应用}
\subsection{深部地质研究内容}
深部地质的研究对象是地球深部各个带的结构、成分和构造,以及它们的发展历史。

地球物理方法所确定的一般是地球内部的圈层或其他不均与介质的几何形状和物理参数。
这也是深部地质学的重要任务,但不是唯一任务。虽然深部的化学成分,不能为地球物理方法所确认,但现阶段的深部研究
起最显著作用的还是地球物理学。
\subsection{一些深部地球物理方法}
目前,地球物理方法依然是研究地壳深部构造的最有效、最准确和最可靠的方法。最常采用的是地震方法、
重力测量法、磁法和电法。
应当指出,各种地球物理方法是目前认识地球深部构造的\emph{唯一}方法。
虽然在实际地质条件下,地球物理存在多解问题,但这是因为地球物理方法的客观性。
\subsubsection{地震方法}
用地震方法研究地壳和地幔的构造、物质状态、应力状态以及岩石圈的其他参数,是通过
对各种地震波的速度、传播特点、能量吸收等资料的分析来进行的。

确定深部构造,是以利用远震或近震的体波为基础的。
根据地震波的走时或地球的或某一地区的平均走时曲线的偏差,可以求出莫霍面的埋藏深度。
如果纵波在地幔顶部的传播速度是稳定的,则时距曲线反映台站到震源的距离$R$与地震波走时$t$之间的关系,其直线段
对应地幔顶部。根据来自地壳或地幔其他面上的地震波,可用类似方法在时距曲线上大致画出直线段,
但它与坐标轴成另外一种倾角。到地壳底面或其他面的深度的变化会引起与直线的时距曲线的偏离。

用地震方法确定地壳厚度的精度,取决于震源深度、震中距离以及地震波到达各个台站的时间的误差。深部界面的非水平状、壳内一些
层面和非均匀层的折射作用、地震波速度在地壳内的变化,都会对精度产生明显影响。
这些都会造成确定地壳底面和壳内及地幔内其他界面深度时精度低的原因。

地震勘探应用最广泛的是折射波法和反射波法。折射波法可以追踪地壳中的平缓界面,确定界面波的速度。
在勘察工作中,这种方法有可能最有效地研究深部构造的一般特点。用这种方法进行相似界面的地震学对比,通常会
得出单一解。
折射波法的前提是界面介质中的波速有相当大的差异,所研究的界面几乎是水平产状,而且介质是均匀的。
离开这些条件,就会导致严重的错误。被研究界面的起伏程度显得特别重要,它不仅会是歪曲所确定的深度,而且可能形成虚假的多层构造图像。
在研究地壳方面,反射波具有广阔的前景。用反射波法可以追踪有复杂几何形状的界面形态,划分多次波速度反转的剖面。
这种方法在研究深部构造方面的详尽程度从原理上说几乎没有限制。
\subsubsection{重力测量法}
重力测量法能提供关于地壳和地幔中质量分布的概念,所以能判定地球内部的构造和成分,在某种程度上能推测出密度不均匀体的埋藏深度。
该方法是以测量位场即重力异常为基础的。
它提供的使命异常场的总体特征,而异常场是由不同规模、形态和埋藏深度的不均匀体的重力作用叠加而形成的。
因此,必须把实测的异常场划分为许多分量场。分量场由不同地质体引起。根据重力测量资料所作各种图件或计算的质量取决于划分异常场的质量。
此任务极其困难,在大多数情况下不会有单一的解答。反演问题,即根据重力异常计算地质体的形状、深度和密度,也具有多解性。
\subsubsection{磁法}
磁法测量能够查明磁性体在深部的分布状况,并可获得有关磁体的磁化强度和某些磁性体的埋藏深度,
有时还能获得有关磁性体的形状,
但是,仍然有一些不清楚的问题,如磁性体界面与地壳和地幔的深部构造有什么关系;认为某些异常起源于数十千米,这究竟合理到什么程度;
在进行定量计算时,在什么样情况下,考虑或不考虑剩余磁化向量的影响才符合实际,等等。
深部地质学的目的是根据地磁资料计算磁性体的磁化强度、顶底面的深度、形状,甚至计算莫霍面的温度。不过,其可靠性尚受质疑,因为
解释是几乎总是假定磁化是均匀的,磁化强度矢量的方向与现代磁场一直,磁性体形状十分简单,不考虑岩石磁性。
\subsubsection{电法}
在地壳深部探测中,电法勘探用于确定第四系或者其他非变质地层的厚度,用于确定地台层状综合体底面的深度,用于查明地壳
和地幔的低阻层。在进行浅部研究时,主要采用各种不同方案的测深法,这些方法是:垂向电测深、偶极电测深、大地电测深、
电场建立法和磁场建立法。
大地电磁测深法和磁变测深法是以地球磁场随时间变化而形成的电场和磁场为基础的。这些测深方法应用广泛。

在大地电磁测深和磁变测深方案或者剖面方案中,频率测深的应用开辟了广阔的前景。
频率测深的结果属于一个比较小的面积和交叉点,不仅可以确定垂直剖面上的不均匀性,而且可以确定水平方向上的不均匀性。频段很宽,
能不同程度地研究电磁剖面。因此,频率测深法可以来解决全球性、区域型和局部性的问题。
\subsubsection{地热}
地热学的主要任务是获取地球热量在地壳和地幔中分布状况的资料。地热探测可以提供地球上不同地区的地热梯度、
热流和热状态的信息。

地球的热场由很多因素决定,如全球性热源及其厚度和分布深度、地壳上部的局部性热源、岩石的热导
率、地壳构造、地球表面的性质、热交换中的对流因素(地下水、火山爆发)等等。
理想的热场:地球均匀,形状规则;某种同意的热源据地球表面上任何一点的距离都是相等的。
实际情况与之的偏离称为地球热场异常而被记录下来。查明热异常并确定异常的原因就是地热探测的任务。

现在,陆上和海上都已经开始热流研究。水可以隔绝温度的四季变化和昼夜变化。这使得水下条件的热流测量
是一件比在陆地上更容易的事情。加之海底没有地下水流,现在洋底热流的研究比大陆上还更加精细。在陆地上,热流只能通过
比水循环带更深的钻孔来测定。近几年研究表明,深度达$200m$的钻孔深度为地热探测的最低要求。
\section{综合地球物理方法在区域性地质填图中的应用}
地质填图是依次研究地下地质结构的过程。在地质勘探工作的所有阶段都进行地质填图,
并以绘制成地质图而结束。
这些地质图就是矿床预测、普查和勘探的基础。

在拟定规划时,地质填图的地质勘探工分为:
\begin{enumerate}
\item 比例尺$1:200000(1:100000)$的区域地球物理调查,随即进行相同比例尺的地质测量;
\item 比例尺$1:50000(1:25000)$的地质测量;
\item 深部地质填图,同时预测开采能力所及深度内的矿产;
\item 比例尺$1:10000(1:2000)$的详细地质填图;
\end{enumerate}
在地质填图的同时,进行相应比例尺的综合地球物理调查,这些工作或者与地质测量同时进行,或者在地质填图前进行。

区域型地质填图的目的是进行构造分区,编制预测图,查明区域内最重要的地质构造特征和矿产分布位置。
综合地球物理方法在地质填图前开展,包括航空方法和地面方法中的地震工作、磁法测量和重力测量以及电法测量。
\subsection{工作的设计}
收集现有的全部地质资料和地球物理资料是设计的基础。另外,要充分分析区域内地球物理场和岩石物理性质。
应当充分注意以前进行过的地球物理测量,和分析邻近区域的工作成果,提出任务,然后转入选择综合方法及合理的野外观测方法。
\subsection{解决的问题}
在区域型地质填图阶段,地球物理综合方法的基本任务是:
\begin{enumerate}
\item 研究地壳整个厚度内的结构,确定地壳上部各层结构与深部构造的关系;
\item 利用有关深部构造的地球物理信息对调查区进行构造分区,并准备预测图的地质构造底图;
\item 研究区域内地质结构特点,为较大比例尺的填图准备地质-地球物理底图,查明矿产分布的规律和划分远景区。
\end{enumerate}
\subsection{方法的综合}
在区域填图时,必要的地球物理调查是:
\begin{enumerate}
\item 比例尺$1:200000$的重磁测量,测量的主要任务是区域性大地构造分区和区域内的的比例尺构造填图;
\item 比例尺$1:200000$的航磁测量,用于区内的岩性填图,并补充研究主要的构造单元,如侵入体、断裂带等。
\end{enumerate}
\subsection{区域性地质调查的地球物理方法}
在区域性调查时,每个地质区都要不知几条深部地震测深测线,结合重磁和电法勘探的观测,
沿测线编制反应地壳整个厚度内的结构的地质\cndash 地球物理基准剖面。
深部地震测深通常与记录近震与远震弹性波的地震观测配合进行。这时,地震测深剖面作为基准骨架,而地震观测填充其中。

在区域性地质调查中,磁法勘探应用最广。磁法勘探为地台区和褶皱区的构造分区提供丰富的资料。可以解决的问题有:
\begin{enumerate}
\item 粗略地圈定具有不同地质发育历史的地壳的大地段(地台、地台型盆地和隆起、地槽);
\item 追索不同时代的褶皱带和查明主要构造单元活动的构造关系;
\item 追索岩墙杂岩体、超基性岩带和查明构造断裂;
\item 查明侵入和喷出岩浆活动有不同程度的表现的地段;
\item 对地质建造的范围填图并解决其他构造分区问题。
\end{enumerate}
在编制地质构造图和大地构造略图时,磁场中各种地质特征的表现形态多种多样。因此应用磁法勘探资料,必须和其他方法结合。
特别是在填绘地台区的褶皱基地是,应用电法勘探、地震勘探和重力勘探资料的结果更能解决问题。

从磁法勘探和重力勘探的资料可以获得有关结晶基地内部褶皱断块构造、基底表面起伏、
基地内巨大断裂带和岩浆建造分布。

在研究结晶基底起伏时,在综合方法中广泛应用电法勘探方法。沉积地层中有高电阻层。
故可以使用大地电磁法、大地电磁剖面法、大地电磁联合剖面法和大地电磁测深法等方法

航空伽玛测量可以大大增加有关侵入岩和沉积岩在出露区分布的信息。
\subsection{资料的整理和解释}
通过在地质测量之前进行的地面地球物理调查可获得该区的各种物理场图和地质构造略图,如:
\begin{enumerate}
\item $Z_a$等值线图和方向互相垂直的$Z_a$曲线平面剖面图
\item 标准中间层密度和换算到海平面的$\Delta g$布格图
\item 各类电测深曲线图、疏松沉积层的等厚度图和标准层顶板地形图
\item 一些元素的等浓度图
\end{enumerate}
沿贯穿工作区主要构造单元的测线还要绘制地质\cndash 地球物理剖面图。
在研究岩石物理性质的基础上,对于具有参数统计特征的所有地层层为要编制表格和变化曲线(直方图)。

要在所研制的填图目标的物理\cndash 地质模型的基础上解释地球物理资料。这时,每个地质事实都应看作没有去掉
误差的单个地质观测。因此,地质结构应该用它们的置信程度来表征。为了阐明置信程度不高的异常的地质性质,要布置检查(填图)钻探。
\section{地球物理方法在工程探测与检测中的应用}
\subsection{工程地质的研究对象和任务}
人类的工程活动与地质环境之间,处于既相互联系又相互制约的矛盾中。研究地质环境与人类工程活动之间的关系,促使两者之间的
矛盾的转化和解决,就是工程地质学的基本任务。工程地质学为工程建设服务,是通过工程地质勘查来实现的。勘查所取得的各项地质资料和数据
,提供给规划、设计、施工等部门使用。工程地质勘查的主要任务是:
\begin{enumerate}
\item 阐明建筑地区的工程地质条件,并指出对建筑物有利的和不利的因素;
\item 论证建筑物所存在的工程地质问题,进行定性和定量的评价,做出确切的结论;
\item 选择地质条件优良的建筑场地,并根据场地工程地质条件对建筑物配置提出建议;
\item 研究建筑物兴建后对地质环境的影响,预测其发展演化趋势,提出利于和保护环境的对策和措施;
\item 根据所选定地点的工程地质条件和存在的工程地质问题,提出有关建筑物类型、规模、结构和施工方法的合理建议,以
保证建筑物的正常施工和使用所要求的地质条件;
\item 对工程质量进行检测,为改善工程质量的措施方案提供工程缺陷分析依据。
\end{enumerate}
\subsection{工程物探解决工程地质问题的特点}
工程物探通过自然地球物理场和人工地球物理场对土层、岩体表面或内部的物性参数进行分析、评价、从而得出沿途的结构、
构造、裂隙、含水量、密度、抗压强度、抗剪强度和放射性等强度作特征,进行解决各种岩土工程问题提供依据的一种勘探方法。
选择物探方法有以下原则:
\begin{enumerate}
\item 选择适当种类信息的物探方法
\begin{itemize}
\item 适用的方法要能测量不同物理场的要素或同一场的不同物理量。
\end{itemize}
\item 工作顺序的确定
\begin{itemize}
\item 严格遵循以提高研究精度为特征的工作顺序。尽可能降低费用,增加信息密度。
\end{itemize}
\item 基本方法与详查方法的合理组合
\begin{itemize}
\item 利用一种或数种基本方法按均匀的测网调查全区。其余的方法作为辅助方法,以较高的详细程度
在个别测线上进行。基本方法尽可能简便、低费用和高效率。
\end{itemize}
\item 应用条件的考虑
\begin{itemize}
\item 充分考虑地形、地貌、干扰及其他因素。
\end{itemize}
\item 地质、物探、钻探相互配合
\item 工程\cndash 经济效益原则
\end{enumerate}
\subsection{工程地质与检测问题所适用的综合物探方法}
在岩土勘察、设计、施工检测中,
通常用综合物探技术查明的主要问题有以下几个方面:
\begin{enumerate}
\item 查明覆盖层厚度及其变化与基岩面的起伏,测定覆盖层的物性参数,进行覆盖层分层;
\item 查明风化层深度和划分风华带;
\item 探测隐伏构造、破碎带和追索断层;
\item 查明水文地质条件,测定水文地质参数;
\item 测定岩土体的物理力学参数;
\item 岩溶、洞穴、废弃地下工程的探测;
\item 滑坡、柔软夹层的探测;
\item 岩土体地基及基础施工检测;
\item 围岩质量检测;
\item 桩基检测、灌浆质量效果检测;
\item 天然建筑材料的勘察;
\item 地下埋设物(古墓、地下管线等)的探测。
\end{enumerate}
\end{document}
